% called by main.tex
%
\chapter{Estado de la cuestión}
\label{ch::capitulo2}

\section{Cómo utilizar esta plantilla}
Esta plantilla se ha desarrollado de acuerdo con las pautas de formato de las tesis doctorales de la UPM, que – en muchos aspectos – siguen estándares de formato comunes a muchas instituciones.

El documento se divide en tres componentes (las páginas iniciales, el cuerpo principal y las páginas finales), cada uno de los cuales tiene varios subcomponentes, como se muestra en la Tabla \ref{tab:thesisComp}:

\begin{table}[h]
\centering
    \caption{Componentes del documento de tesis}
    \label{tab:thesisComp}
\begin{tabularx}{\textwidth}{l l}
\toprule
\textbf{Componentes} & \textbf{Subcomponentes}\\
\midrule
\multirow{9}{10em}{Páginas iniciales} & Cubierta \\ 
 & Portada \\
  & Página de créditos \\
  & [Dedicatoria, Agradecimientos] \\
    & Abstract/Resumen \\
     & Tabla de Contenido \\
     & Lista de Figuras \\
     & Lista de Tablas \\
     & Abreviaturas y Acrónimos \\
\midrule 
\multirow{3}{10em}{Cuerpo principal} & Introducción \\ 
 & Capítulos centrales específicos de cada tesis \\
 & Conclusiones \\
\midrule
\multirow{2}{10em}{Páginas finales} & Referencias \\ 
     & Anexos\\
\bottomrule
    \end{tabularx}
\end{table}

La mayor parte de normas de formato se refieren a las páginas iniciales, que son las que aparecen antes de los capítulos de la tesis (el cuerpo principal de la tesis).

Las normas de obligado cumplimiento y algunas recomendaciones se presentan a continuación. 

\section{Normas de obligado cumplimiento}
Las \textbf{normas clave} a recordar son:
\begin{itemize}
    \item Las tres primeras páginas (cubierta, portada y página de créditos) deben utilizar el formato especificado en esta plantilla. Si el programa de doctorado es interinstitucional la cubierta puede variar (verificar la plantilla a utilizar en este caso).
    \item Las páginas iniciales son obligatorias, a excepción de dedicatoria y agradecimientos.
    \item La tesis se presenta en formato electrónico (PDF).
    \item Idioma: la tesis puede redactarse en español o en inglés.
    \item Tamaño de página: A4
    \item Numeración: Las páginas iniciales – antes del comienzo de los capítulos – utilizan números romanos centrados en el margen inferior, excluyendo la cubierta y la portada, que no deben numerarse. El resto del texto (a partir de la Introducción) utiliza números arábigos centrados en la parte inferior.
\end{itemize}

La \textbf{cubierta}:
\begin{itemize}
    \item Debe elaborarse utilizando esta plantilla, introduciendo los datos oportunos en los apartados marcados con < > y sin incluir otra información.
    \item Debe incluir el nombre oficial del centro, en español y sin abreviar (ver \href{https://www.upm.es/sfs/Rectorado/Vicerrectorado%20de%20Investigacion/Doctorado/Tesis/impresos/5_Listado%20de%20PD%20y%20sus%20centros.pdf}{Listado}).
    \item Debe incluir el nombre completo del doctorando.
\end{itemize}

La \textbf{portada}:
\begin{itemize}
    \item Debe elaborarse utilizando esta plantilla, introduciendo los datos oportunos en los apartados marcados con < > y sin incluir otra información.
    \item Debe incluir el nombre oficial del centro, en español y sin abreviar (ver \href{https://www.upm.es/sfs/Rectorado/Vicerrectorado%20de%20Investigacion/Doctorado/Tesis/impresos/5_Listado%20de%20PD%20y%20sus%20centros.pdf}{Listado}). Puede incluir el logo del centro, respetando el tamaño marcado en la plantilla (ver \href{https://www.upm.es/UPM/SalaPrensa/IdentidadGrafica/LogosPlantillas}{Logos}).
    \item Debe incluir el nombre completo del programa de doctorado (ver \href{https://www.upm.es/sfs/Rectorado/Vicerrectorado%20de%20Investigacion/Doctorado/Tesis/impresos/5_Listado%20de%20PD%20y%20sus%20centros.pdf}{Listado}).
    \item Debe incluir el nombre completo del doctorando.
    \item Debe incluir el nombre completo del director de tesis (y del codirector si lo hubiese).
\end{itemize}

La \textbf{página de créditos} (página i):
\begin{itemize}
    \item Debe elaborarse utilizando esta plantilla, introduciendo los datos oportunos en los apartados marcados con < > y sin incluir otra información.
    \item Incluye datos adicionales del director de tesis: puesto e institución (lo mismo para el codirector si lo hubiese).
    \item Los apartados “Revisores Externos”, “Tribunal de tesis” y “Fecha de defensa” se dejan sin completar.
    \item Si la tesis ha recibido financiación de alguna convocatoria competitiva, debe indicarse al final de esta página (ver plantilla).
\end{itemize}

El \textbf{resumen}:
\begin{itemize}
    \item Se debe incluir un resumen tanto en español como en inglés, independientemente del idioma de la tesis.
    \item Formato: máximo de 4000 caracteres, texto plano (sin símbolos)
    \item Estructura: el resumen es una presentación de la tesis y, por tanto, debe tener una estructura clara, incluyendo introducción (o motivación), objetivos, desarrollo y conclusiones.
\end{itemize}

\section{Recomendaciones}
Por lo general, el cuerpo principal de la tesis abarca varios capítulos, cuyo número y estructura puede variar dependiendo del ámbito de conocimiento y de si se trata de un manuscrito o de una tesis por compendio de publicaciones. A título orientativo, en esta plantilla se incluyen los siguientes capítulos: Introducción, Estado de la cuestión, Material y métodos, Resultados, Discusión y Conclusiones. La estructura del cuerpo principal de cada tesis específica debe consultarse con los directores de tesis. 

Lo ideal es que cada capítulo esté organizado en secciones y subsecciones, con títulos numerados.

En cuanto a la fuente, los estilos recomendados son Times New Roman, Century, Arial, Book Antiqua, o similares.

\begin{figure}[h]
\centering
    \includegraphics[width=0.7\textwidth]{figures/latexthesis.png}
\caption{Escribiendo tu tesis}
\label{phd1}
\end{figure}

Utilizar una plantilla ayuda a utilizar el mismo formato a lo largo de toda la tesis. Es aconsejable utilizar siempre el mismo estilo para incluir citas en el texto, como también recomiendan otros autores (\cite{bellDoingYourResearch2010}, \cite{carterIgnoringMePart2017}, \cite{odenaHowDoctoralStudents2017}, \cite{riveracaminoComoEscribirPublicar2014}), para dar formato a las tablas (ver Tabla 1) y para el formato de las figuras (ver Figura 1). Se recomienda numerar las tablas y figuras por capítulos (Tabla 1.1, etc.).

La parte final de la tesis comprende las referencias y los anexos.

Las referencias deben incluirse utilizando el estilo recomendado para cada ámbito de conocimiento. Esta plantilla utiliza APA, pero cada tesis deberá utilizar el estilo estándar en su ámbito de conocimiento.

Los anexos incluyen material adicional no incluido en el texto principal (cuestionarios, resultados adicionales, etc.). Se recomienda la numeración alfabética de los anexos (A, B, ...) y comenzar cada anexo en una página distinta. En el caso de incluir tablas en anexos, se iniciará una nueva numeración independiente de la del cuerpo de la tesis (Tabla A.1., etc.).





