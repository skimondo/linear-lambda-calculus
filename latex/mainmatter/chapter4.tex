% called by main.tex
%
\chapter{Travaux réalisés}
\label{sec::chapitre4}

\section{La logique linéaire lambda dans Coq}

Ce projet vise à traduire une implémentation du lambda-calcul linéaire de l'assistant de preuve Beluga à Coq. Une implémentation basée sur la méthode CARVe dans Beluga est disponible sur Zenodo :
\url{https://zenodo.org/records/14271731}.

Le dépôt du projet réalisé, qui implémente la méthode CARVe dans l'assistant de preuve Coq, est disponible sur GitHub : \url{https://github.com/skimondo/linear-lambda-calculus}.

L'objectif était de comparer les deux assistants de preuve en termes de complexité des preuves, des défis d'implémentation et de leur facilité d'utilisation globale. Le fichier \texttt{comparison.md} dans le dépôt détaille toutes les différences rencontrées lors de la traduction des définitions et des lemmes.

\section{Preuves}

L'implémentation des preuves inclut :
\begin{itemize}
    \item La gestion des contextes linéaires, en utilisant des mécanismes explicites pour l'allocation des ressources et la fusion des contextes.
    \item Les propriétés fondamentales comme l'égalité des multiplicités et les règles de fusion des contextes.
    \item La formalisation des définitions inductives pour gérer les contextes et les multiplicités dans les deux assistants de preuve.
\end{itemize}

Le dépôt contient deux répertoires principaux : \texttt{beluga/} et \texttt{coq/}, représentant respectivement les implémentations dans Beluga et Coq. Chaque répertoire inclut des définitions communes et des lemmes illustrant la logique linéaire.

\section{Comparaison entre Beluga et Coq}

Les similarités et différences entre Beluga et Coq pour l'implémentation du lambda-calcul linéaire sont analysées dans le fichier détaillé disponible ici : \url{https://github.com/skimondo/linear-lambda-calculus/blob/main/docs/comparison.md}.

\subsection{Résumé}

\begin{itemize}
    \item \textbf{Similarités} : Les deux assistants de preuve partagent des structures similaires pour les définitions de types, les règles inductives et la gestion des contextes linéaires.
    \item \textbf{Différences} :
    \begin{itemize}
        \item Beluga utilise des contextes implicites et des preuves construites par appariement dépendant.
        \item Coq requiert une gestion explicite des contextes et des preuves guidées par des tactiques.
    \end{itemize}
    \item \textbf{Automatisation des preuves} : Beluga favorise une approche constructive par récursion, tandis que Coq s'appuie sur une construction pas à pas à l'aide de tactiques.
    \item \textbf{Suitabilité} : Beluga est spécialisé pour les systèmes dépendamment typés, tandis que Coq offre une utilité plus générale pour divers cadres de formalisation.
\end{itemize}
