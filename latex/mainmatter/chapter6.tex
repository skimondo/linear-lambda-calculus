% called by main.tex
%
\chapter{Conclusion}
\label{sec::chapitre6}

\section{Idées clés}

Le projet a mis en évidence l'importance de la logique linéaire comme fondement pour modéliser des systèmes où la gestion rigoureuse des ressources est cruciale. Il a permis de comparer deux outils majeurs, Beluga et Coq, dans leur capacité à encoder et à manipuler la logique linéaire à travers des cadres formels.

\section{Résultats}

Les principaux résultats de ce projet incluent :
\begin{itemize}
    \item Une implémentation du lambda-calcul linéaire dans Coq, disponible sur GitHub : \url{https://github.com/skimondo/linear-lambda-calculus}.
    \item L'encodage des contextes linéaires en utilisant la méthode CARVe pour gérer les ressources de manière explicite et rigoureuse.
    \item Une analyse comparative des assistants de preuve Beluga et Coq, détaillant leurs forces et leurs limites respectives.
\end{itemize}

\section{Récapitulatif}

Ce projet a démontré que la logique linéaire est un outil puissant pour modéliser et raisonner sur des systèmes formels complexes. En traduisant une implémentation de Beluga à Coq, nous avons identifié des approches pour manipuler les contextes et prouver des propriétés essentielles des systèmes linéaires.

\section{Travaux futurs}

\subsection{Librairie de preuves pour le lambda-calcul linéaire}

Un objectif futur est de développer une librairie complète de preuves pour le lambda-calcul linéaire dans Coq. Cette librairie inclura des mécanismes automatisés pour vérifier les propriétés des programmes et des contextes, facilitant ainsi l'usage pratique de la logique linéaire dans des applications réelles.

\subsection{Exploration de méthodes alternatives}

Il serait pertinent d'explorer des méthodes alternatives à CARVe pour encoder la logique linéaire dans Coq ou d'autres assistants de preuve. Ces méthodes pourraient offrir des perspectives différentes ou résoudre certains des défis spécifiques rencontrés avec CARVe. Cette diversification permettrait d'élargir les options disponibles pour la mécanisation de la logique linéaire et d'enrichir les outils théoriques à disposition.
