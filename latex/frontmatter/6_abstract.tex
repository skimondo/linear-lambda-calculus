% called by main.tex
%

\section*{Résumé}
\addcontentsline{toc}{section}{Résumé}
\label{sec::resume}

Cette étude explore l'encodage de la logique linéaire dans l'assistant de preuve Coq, en répondant au défi de formaliser les logiques sous-structurales. Les objectifs incluent la définition d'un cadre clair pour l'encodage des règles logiques, la mise à l'épreuve des fonctionnalités de Coq et la démonstration de la praticité de cette approche à travers des exemples. Une technique novatrice, la méthode CARVe, introduite par un collaborateur, a permis d'améliorer l'efficacité de l'encodage des propositions de logique linéaire. Les résultats contribuent à faire progresser les méthodes formelles dans les assistants de preuve, avec des implications importantes pour les langages de programmation quantique.


